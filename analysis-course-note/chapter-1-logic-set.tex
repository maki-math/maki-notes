
\section{数理逻辑与集合论}
这一章作为高等数学的基础,是非常重要的。请至少做出80\%的习题,保证对后面章节而言能有充分的知识储备。

\subsection{命题}
maki注:一个命题要么是真的,要么是假的。命题不能同时是真的和同时是假的,或者同时不真和同时不假。

证明$\neg p \iff$ 证伪 $p$。

证伪$\neg p \iff$ 证明 $p$。

$p$是真命题 $\iff$ $p$。

$p$是假命题 $\iff$ $\neg p$。此时我们也可以说非$p$是真的($\neg p = T$)。

\subsection{与或非和二元、一元运算}

我们使用真值表演示n元运算的结果。
真值表通过枚举n元运算中每一个变量的值(即T/F),演示每一组值经过n元运算后的逻辑结果。

zyc注:二元运算和一元运算并不只包含与或非。你也可以定义自己的n元运算,当且仅当以任何方式定义运算的完整真值表。

\subsubsection{与}

\begin{table}[H]
    \begin{tabular}{ll|l}
    $p$ & $q$   & $p \wedge q$ \\ \hline 
    T   & T     & T            \\
    T   & F     & F            \\
    F   & T     & F            \\
    F   & F     & F                        
    \end{tabular}
\end{table}

\subsubsection{或}

\begin{table}[H]
    \begin{tabular}{ll|l}
    $p$ & $q$   & $p \vee q$ \\ \hline 
    T   & T     & T            \\
    T   & F     & T            \\
    F   & T     & T            \\
    F   & F     & F                        
    \end{tabular}
\end{table}

\subsubsection{非}

\begin{table}[H]
    \begin{tabular}{l|l}
    $p$ & $\neg p$   \\ \hline 
    T   & F     \\
    F   & T                        
    \end{tabular}
\end{table}

\subsection{与或非的运算定律}

\subsubsection{交换律}

\begin{equation}
p \wedge q \iff q \wedge p
\end{equation}

\begin{equation}
p \vee q \iff q \vee p
\end{equation}

\subsubsection{结合律}

\begin{equation}
    p \wedge (q \wedge r) \iff (p \wedge q) \wedge r
\end{equation}

\begin{table}[H]
    \begin{tabular}{lll|l|l|l|l}
    p & q & r & $q \wedge r$ & $p \wedge (q \wedge r)$ & $p \wedge q$ & $(p \wedge q) \wedge r$ \\ \hline
    T & T & T & T            & T                       & T            & T                       \\
    T & T & F & F            & F                       & T            & F                       \\
    T & F & T & F            & F                       & F            & F                       \\
    T & F & F & F            & F                       & F            & F                       \\
    F & T & T & T            & F                       & F            & F                       \\
    F & T & F & F            & F                       & F            & F                       \\
    F & F & T & F            & F                       & F            & F                       \\
    F & F & F & F            & F                       & F            & F                      
    \end{tabular}
\end{table}

\begin{equation}
    p \vee (q \vee r) \iff (p \vee q) \vee r
\end{equation}

\begin{table}[H]
    \begin{tabular}{lll|l|l|l|l}
    p & q & r & $q \vee r$ & $p \vee (q \vee r)$ & $p \vee q$ & $(p \vee q) \vee r$ \\ \hline
    T & T & T & T            & T                       & T            & T                       \\
    T & T & F & T            & T                       & T            & T                       \\
    T & F & T & T            & T                       & T            & T                       \\
    T & F & F & F            & T                       & T            & T                       \\
    F & T & T & T            & T                       & T            & T                       \\
    F & T & F & T            & T                       & T            & T                       \\
    F & F & T & T            & T                       & F            & T                       \\
    F & F & F & F            & F                       & F            & F                      
    \end{tabular}
\end{table}

\subsubsection{分配律}

maki注:此处类比$a \cdot (b + c) = a \cdot b + a \cdot c$。

\begin{equation}
    p \wedge (q \vee r) \iff (p \vee q) \wedge (p \vee r)
\end{equation}

\newpage