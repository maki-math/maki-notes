
\section{数理逻辑与集合论}
这一章作为高等数学的基础,是非常重要的。请至少做出80\%的习题,保证对后面章节而言能有充分的知识储备。

\subsection{命题}
maki注:一个命题要么是真的,要么是假的。命题不能同时是真的和同时是假的,或者同时不真和同时不假。

证明$\neg p \iff$ 证伪 $p$。

证伪$\neg p \iff$ 证明 $p$。

$p$是真命题 $\iff$ $p$。

$p$是假命题 $\iff$ $\neg p$。此时我们也可以说非$p$是真的($\neg p = T$)。

\subsection{与或非和二元、一元运算}

我们使用真值表演示n元运算的结果。
真值表通过枚举n元运算中每一个变量的值(即T/F),演示每一组值经过n元运算后的逻辑结果。

zyc注:二元运算和一元运算并不只包含与或非。你也可以定义自己的n元运算,当且仅当以任何方式定义运算的完整真值表。

\subsubsection{与}

\begin{table}[H]
    \begin{tabular}{ll|l}
    $p$ & $q$   & $p \wedge q$ \\ \hline 
    T   & T     & T            \\
    T   & F     & F            \\
    F   & T     & F            \\
    F   & F     & F                        
    \end{tabular}
\end{table}

\subsubsection{或}

\begin{table}[H]
    \begin{tabular}{ll|l}
    $p$ & $q$   & $p \vee q$ \\ \hline 
    T   & T     & T            \\
    T   & F     & T            \\
    F   & T     & T            \\
    F   & F     & F                        
    \end{tabular}
\end{table}

\subsubsection{非}

\begin{table}[H]
    \begin{tabular}{l|l}
    $p$ & $\neg p$   \\ \hline 
    T   & F     \\
    F   & T                        
    \end{tabular}
\end{table}

\subsection{与或非的运算定律}

\subsubsection{交换律}

\begin{equation}
p \wedge q \iff q \wedge p
\end{equation}

\begin{equation}
p \vee q \iff q \vee p
\end{equation}

\subsubsection{结合律}

\begin{equation}
    p \wedge (q \wedge r) \iff (p \wedge q) \wedge r
\end{equation}

\begin{table}[H]
    \begin{tabular}{lll|l|l|l|l}
    p & q & r & $q \wedge r$ & $p \wedge (q \wedge r)$ & $p \wedge q$ & $(p \wedge q) \wedge r$ \\ \hline
    T & T & T & T            & T                       & T            & T                       \\
    T & T & F & F            & F                       & T            & F                       \\
    T & F & T & F            & F                       & F            & F                       \\
    T & F & F & F            & F                       & F            & F                       \\
    F & T & T & T            & F                       & F            & F                       \\
    F & T & F & F            & F                       & F            & F                       \\
    F & F & T & F            & F                       & F            & F                       \\
    F & F & F & F            & F                       & F            & F                      
    \end{tabular}
\end{table}

\begin{equation}
    p \vee (q \vee r) \iff (p \vee q) \vee r
\end{equation}

\begin{table}[H]
    \begin{tabular}{lll|l|l|l|l}
    p & q & r & $q \vee r$ & $p \vee (q \vee r)$ & $p \vee q$ & $(p \vee q) \vee r$ \\ \hline
    T & T & T & T            & T                       & T            & T                       \\
    T & T & F & T            & T                       & T            & T                       \\
    T & F & T & T            & T                       & T            & T                       \\
    T & F & F & F            & T                       & T            & T                       \\
    F & T & T & T            & T                       & T            & T                       \\
    F & T & F & T            & T                       & T            & T                       \\
    F & F & T & T            & T                       & F            & T                       \\
    F & F & F & F            & F                       & F            & F                      
    \end{tabular}
\end{table}

\subsubsection{分配律}

maki注:此处类比$a \cdot (b + c) = a \cdot b + a \cdot c$。

\begin{equation}
    p \wedge (q \vee r) \iff (p \vee q) \wedge (p \vee r)
\end{equation}

\subsection{蕴含和推导}

$\implies$ 实质推导。该符号表示推导的二元运算。

$\implies$ 语义推导 

\begin{table}[H]
    \begin{tabular}{ll|l}
    $p$ & $q$   & $p \implies q$ \\ \hline 
    T   & T     & T            \\
    T   & F     & F            \\
    F   & T     & T            \\
    F   & F     & T                        
    \end{tabular}
\end{table}

\begin{theorem}
    $(p \implies q) \iff (\neg q \implies \neg p)$
\end{theorem}


\begin{proof}
    使用反证法证明。
\begin{equation}
    \begin{split}
        & (\neg q \implies \neg p) \\
        \iff & \neg (\neg q) \vee \neg p \\
        \iff & q \vee \neg p \\
        \iff & \neg p \vee q \\
        \iff & (p \implies q)
    \end{split}
\end{equation}
\end{proof}

maki注:反证法 - 若要证明p,可以证伪非p,即假设p为假,则可以推得矛盾,以此证明p不可能为假。

\begin{theorem}
    $((p \vee q) \implies r) \iff (p \implies r) \wedge (q \implies r)$
\end{theorem}

\begin{proof}
左右相等
\begin{equation}
    \begin{split}
        LHS \iff & (\neg (p \vee q)) \vee r \\
        \iff & (\neg p \wedge \neg q) \vee r \\
        \iff & (\neg p \vee r) \wedge (\neg q \vee r) \\
        \iff & (p \implies r) \wedge (q \implies r)
    \end{split}
\end{equation}
\end{proof}

\subsection{存在和所有量词}

$\forall$ 对于所有

\begin{example}
    $\forall x \in \mathbb{R}. x^2 \ge 0$
\end{example}

\begin{proof}
令$x$为实数。$x^2 \ge 0$ 恒成立。证毕。
\end{proof}


\begin{example}
    $\forall x \in \mathbb{R}. x^2 \ge 0$
\end{example}

\begin{proof}
令$x$为实数。$x^2 \ge 0$ 恒成立。证毕。
\end{proof}


\begin{example}
    $\forall x \in \mathbb{R}. \forall y \in \mathbb{R}. x^2 + y^2 \ge 2xy$
\end{example}

\begin{proof}
令$x$为实数。令$y$为实数。通过例1.1,得到$(x-y)^2 \ge 0$.
\begin{equation}
    \begin{split}
        & (x-y)^2 \ge 0 \\
        \iff & x^2 + y^2 - 2xy \ge 0 \\
        \iff & x^2 + y^2 \ge 2xy
    \end{split}
\end{equation}
得到$x^2 + y^2 \ge 2xy$,证毕。
\end{proof}

$\exists$ 存在一个

\begin{example}
    $\forall x \in \mathbb{R}. x^2 \ge 0$
\end{example}

\begin{proof}
令$x$为实数。$x^2 \ge 0$ 恒成立。证毕。
\end{proof}

\begin{example}
    $\exists x \in \mathbb{N}_1. \exists y \in \mathbb{N}_1. x + y = 3$
\end{example}

\begin{proof}
令$x$为$1$。令$y$为$2$。$x + y = 1 + 2 = 3$。证毕。
\end{proof}

量词的否定

\begin{equation}
    \neg (\forall x \in A. p(x)) \iff (\exists x \in A. \neg p(x))
\end{equation}
\begin{equation}
    \neg (\exists x \in A. p(x)) \iff (\forall x \in A. \neg p(x))
\end{equation}

maki注:量词颠倒,命题取非。

\begin{example}
    $\neg (\exists x \in A. \exists y \in B. \forall z \in C. p(x, y, z))$
\end{example}

\begin{proof}
    \begin{equation}
        \begin{split}
            & \neg (\exists x \in A. \exists y \in B. \forall z \in C. p(x, y, z)) \\
            \iff & \forall x \in A. \neg (\exists y \in B. \forall z \in C. p(x, y, z)) \\
            \iff & \forall x \in A. \forall y \in B. \neg (\forall z \in C. p(x, y, z)) \\
            \iff & \forall x \in A. \forall y \in B. \exists z \in C. \neg (p(x, y, z)) 
        \end{split}
    \end{equation}
\end{proof}


\newpage